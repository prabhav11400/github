\documentclass{beamer}
\usetheme{CambridgeUS}

\setbeamertemplate{caption}[numbered]{}

\usepackage{enumitem}
\usepackage{tfrupee}
\usepackage{amsmath}
\usepackage{amssymb}
\usepackage{gensymb}
\usepackage{graphicx}
\usepackage{txfonts}

\def\inputGnumericTable{}

\usepackage[latin1]{inputenc}                                 
\usepackage{color}                                            
\usepackage{array}                                            
\usepackage{longtable}                                        
\usepackage{calc}                                             
\usepackage{multirow}                                         
\usepackage{hhline}                                           
\usepackage{ifthen}
\usepackage{caption} 
\captionsetup[table]{skip=3pt}  
\providecommand{\pr}[1]{\ensuremath{\Pr\left(#1\right)}}
\providecommand{\cbrak}[1]{\ensuremath{\left\{#1\right\}}}
\renewcommand{\thefigure}{\arabic{table}}
\renewcommand{\thetable}{\arabic{table}}                                     
                               
\title{AI1110 \\ Assignment 11}
\author{Ankit Saha \\ AI21BTECH11004}
\date{18 May 2022}


\begin{document}
	% The title page
	\begin{frame}
		\titlepage
	\end{frame}
	
	% The table of contents
	\begin{frame}{Outline}
    		\tableofcontents
	\end{frame}
	
	% The question
	\section{Question}
	\begin{frame}{CBSE Probability Grade 12 Exercise 13.5.13}
	It is known that $10\%$ of certain articles manufactured are defective. What is the probability that in a random sample of $12$ such articles, $9$ are defective?
	\end{frame}
	
	% The solution
	\section{Solution}
	\begin{frame}{Solution}
	Let a Bernoulli random variable $X \in \cbrak{0,1}$ denote whether the chosen sample is defective or not.
	\begin{table}[ht!]
		\centering
		\begin{tabular}{|l|c|c|}

\hline
\textbf{Parameter} & \textbf{Symbol/Formula} & \textbf{Value} \\
\hline
Total investment & $P$ & $4500$ \\
\hline
Face value of a share & $F$ & $100$ \\
\hline
Discount on shares & $d$ & $10$ \\
\hline
Dividend & $D$ & $7.5$ \\
\hline
Number of shares & $N = \dfrac{100P}{F(100-d)}$ & ??? \\
\hline
Annual income & $A = \dfrac{PD}{100-d}$ & ??? \\
\hline

\end{tabular}
		\caption{Bernoulli distribution}
		\label{table:table1}	
	\end{table}
	\end{frame}
		
	\begin{frame}
	Consider an experiment consisting of $12$ Bernoulli trials and denote the number of defective samples obtained by a binomial random variable $Y \in \cbrak{0,1,\ldots,12}$. This can be expressed as a binomial distribution with probability mass function given by:
	\begin{align}
		p_Y(k) = \binom{n}{k} (1-p)^{n-k} p^k,~ 0 \le k \le n
	\end{align}
	where $n = 12$ and $p = 0.1$
	\end{frame}

	% The answer
	\section{Answer}
	\begin{frame}{Answer}
	The desired probability is given by:
	\begin{align}
		p_Y(9)&= \binom{12}{9} (1-0.1)^3 (0.1)^9 \\
		&\approx 1.6 \times 10^{-7}
	\end{align}
	\end{frame}
	
	% The plot
	\section{Plot}
	\begin{frame}{Plot of the probability mass function}
		\centering
		\includegraphics[height=0.8\paperheight]{figs/fig-1.png}
	\end{frame}
	
\end{document}
