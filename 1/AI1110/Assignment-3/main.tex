\documentclass[journal,12pt,twocolumn]{IEEEtran}

\usepackage{enumitem}
\usepackage{amsmath}
\usepackage{amssymb}
\usepackage{gensymb}
\usepackage{graphicx}

\def\inputGnumericTable{}

\usepackage[latin1]{inputenc}                                 
\usepackage{color}                                            
\usepackage{array}                                            
\usepackage{longtable}                                        
\usepackage{calc}                                             
\usepackage{multirow}                                         
\usepackage{hhline}                                           
\usepackage{ifthen}
\usepackage{caption} 
\captionsetup[table]{skip=3pt}  

\renewcommand{\thefigure}{\arabic{table}}
\renewcommand{\thetable}{\arabic{table}}                                     
                               
\title{Assignment 3 \\ \Large AI1110: Probability and Random Variables \\ \large Indian Institute of Technology Hyderabad}
\author{Ankit Saha \\ \normalsize AI21BTECH11004 \\ \vspace*{20pt} \Large CBSE Statistics Grade 9}


\begin{document}
	% The title
	\maketitle
	
	% The question
	\textbf{Exercise 14.1.1} 
	Give five examples of data that you can collect from your day-to-day life.
	
	% The answer
	\textbf{Answer.}		
	\begin{enumerate}[label=\roman*)]
	\item Number of students from each department taking this course
	\begin{table}[ht!]
		\centering
		\begin{tabular}{|l|c|c|}

\hline
\textbf{Parameter} & \textbf{Symbol/Formula} & \textbf{Value} \\
\hline
Total investment & $P$ & $4500$ \\
\hline
Face value of a share & $F$ & $100$ \\
\hline
Discount on shares & $d$ & $10$ \\
\hline
Dividend & $D$ & $7.5$ \\
\hline
Number of shares & $N = \dfrac{100P}{F(100-d)}$ & ??? \\
\hline
Annual income & $A = \dfrac{PD}{100-d}$ & ??? \\
\hline

\end{tabular}
		\caption{}
		\label{table:table1}	
	\end{table}
	
	\item Maximum temperatures in Hyderabad for the past one week
	\begin{table}[ht!]
		\centering
		\input{tables/table-2.tex}
		\caption{}
		\label{table:table2}	
	\end{table}
	
	\item Average rainfall in Hyderabad during monsoon
	\begin{table}[ht!]
		\centering
		\input{tables/table-3.tex}
		\caption{}
		\label{table:table3}	
	\end{table}
	
	\item Number of students in various programs offered by the institute
	\begin{table}[ht!]
		\centering
		\input{tables/table-4.tex}
		\caption{}
		\label{table:table4}	
	\end{table}
	
	
	\item Number of medals won by India at the 2020 Summer Olympics
	\begin{table}[ht!]
		\centering
		\input{tables/table-5.tex}
		\caption{}
		\label{table:table5}	
	\end{table}
	
	\end{enumerate}
	
	\begin{figure}[!ht]
		\centering
		\includegraphics[width=\columnwidth]{figs/fig-1.png}
		\label{fig1}
	\end{figure}
	
	\begin{figure}[!ht]
		\centering
		\includegraphics[width=\columnwidth]{figs/fig-2.png}
		\label{fig2}
	\end{figure}
	
	\begin{figure}[!ht]
		\centering
		\includegraphics[width=\columnwidth]{figs/fig-3.png}
		\label{fig3}
	\end{figure}
	
	\begin{figure}[!ht]
		\centering
		\includegraphics[width=\columnwidth]{figs/fig-4.png}
		\label{fig4}
	\end{figure}
	
	\begin{figure}[!ht]
		\centering
		\includegraphics[width=\columnwidth]{figs/fig-5.png}
		\label{fig5}
	\end{figure}
	
\end{document}