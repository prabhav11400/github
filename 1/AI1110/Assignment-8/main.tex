\documentclass[journal,12pt,twocolumn]{IEEEtran}

\usepackage{enumitem}
\usepackage{tfrupee}
\usepackage{amsmath}
\usepackage{amssymb}
\usepackage{gensymb}
\usepackage{graphicx}
\usepackage{txfonts}

\def\inputGnumericTable{}

\usepackage[latin1]{inputenc}                                 
\usepackage{color}                                            
\usepackage{array}                                            
\usepackage{longtable}                                        
\usepackage{calc}                                             
\usepackage{multirow}                                         
\usepackage{hhline}                                           
\usepackage{ifthen}
\usepackage{caption} 
\captionsetup[table]{skip=3pt}  
\providecommand{\pr}[1]{\ensuremath{\Pr\left(#1\right)}}
\providecommand{\cbrak}[1]{\ensuremath{\left\{#1\right\}}}
%\renewcommand{\thefigure}{\arabic{table}}
\renewcommand{\thetable}{\arabic{table}}                                     
                               
\title{Assignment 8 \\ \Large AI1110: Probability and Random Variables \\ \large Indian Institute of Technology Hyderabad}
\author{Ankit Saha \\ \normalsize AI21BTECH11004 \\ \vspace*{20pt} \Large CBSE Probability Grade 12}


\begin{document}
	% The title
	\maketitle
	
	% The question
	\textbf{Exercise 13.1.6} 
	A coin is tossed three times. Determine $\pr{E|F}$ for the following three cases: 
	\begin{enumerate}[label=(\roman*)]
	\item
	$E$: head on third toss
	
	$F$: heads on first two tosses
	
	\item
	$E$: at least two heads
	
	$F$: at most two heads
	
	\item
	$E$: at most two tails
	
	$F$: at least one tail
	\end{enumerate}
	
	% The answer
	\textbf{Solution.}
	Let a Bernoulli random variable $X \in \cbrak{0,1}$ denote the possible outcomes of a coin toss.
	\begin{table}[ht!]
		\centering
		\begin{tabular}{|l|c|c|}

\hline
\textbf{Parameter} & \textbf{Symbol/Formula} & \textbf{Value} \\
\hline
Total investment & $P$ & $4500$ \\
\hline
Face value of a share & $F$ & $100$ \\
\hline
Discount on shares & $d$ & $10$ \\
\hline
Dividend & $D$ & $7.5$ \\
\hline
Number of shares & $N = \dfrac{100P}{F(100-d)}$ & ??? \\
\hline
Annual income & $A = \dfrac{PD}{100-d}$ & ??? \\
\hline

\end{tabular}
		\caption{Bernoulli distribution}
		\label{table:table1}	
	\end{table}
	
	Consider an experiment consisting of $3$ Bernoulli trials $X_1, X_2, X_3$ and denote the number of heads obtained by a binomial random variable $Y \in \cbrak{0,1,2,3}$. This can be expressed as a binomial distribution with probability mass function given by:
	\begin{align}
		p_Y(k) = \binom{n}{k} (1-p)^{n-k} p^k,~ 0 \le k \le n
	\end{align}
	where $n = 3$ and $p = \frac12$
	\begin{table}[ht!]
		\centering
		\input{tables/table-2.tex}
		\caption{Binomial distribution}
		\label{table:table2}	
	\end{table}
	
	\begin{enumerate}[label=(\roman*)]
	\item Since the outcome of every coin toss is independent of the outcomes of all preceding coin tosses, $E$ and $F$ are independent events, i.e.,
	\begin{align}
		\pr{E|F} &= \pr{E} \\
		&= \pr{X_3 = 1} \\
		&= p \\
		&= \frac12 = 0.5
	\end{align}
	
	\item
	\begin{align}
		\pr{E|F} &= \pr{Y \ge 2 | Y \le 2} \\
		&= \frac{\pr{Y \ge 2 , Y \le 2}}{\pr{Y \le 2}} \\
		&= \frac{\pr{Y = 2}}{\sum_{i=0}^2 \pr{Y = i}} \\
		&= \frac{\frac38}{\frac18 + \frac38 + \frac38} \\
		&= \frac37 \approx 0.429
	\end{align}
	
	\item
	\begin{align}
		\pr{E|F} &= \pr{Y \ge 1 | Y \le 2} \\
		&= \frac{\pr{Y \ge 1 , Y \le 2}}{\pr{Y \le 2}} \\
		&= \frac{\pr{1 \le Y \le 2}}{\pr{Y \le 2}} \\
		&= \frac{\sum_{i=1}^2 \pr{Y = i}}{\sum_{i=0}^2 \pr{Y = i}} \\
		&= \frac{\frac38 + \frac38}{\frac18 + \frac38 + \frac38} \\
		&= \frac67 \approx 0.857
	\end{align}
	\end{enumerate}	
	
	\begin{figure}[!ht]
		\centering
		\includegraphics[width=\columnwidth]{figs/fig-1.png}
		\caption{Plot of the probability mass function}
		\label{fig1}
	\end{figure}
	
\end{document}