\documentclass{beamer}
\usetheme{CambridgeUS}

\setbeamertemplate{caption}[numbered]{}

\usepackage{amsmath}
\usepackage{amssymb}
\usepackage{gensymb}
\usepackage{graphicx}
\usepackage{txfonts}

\def\inputGnumericTable{}

\usepackage[latin1]{inputenc}                                 
\usepackage{color}                                            
\usepackage{array}                                            
\usepackage{longtable}                                        
\usepackage{calc}                                             
\usepackage{multirow}                                         
\usepackage{hhline}                                           
\usepackage{ifthen}
\usepackage{caption} 
\captionsetup[table]{skip=3pt}

\providecommand{\pr}[1]{\ensuremath{\Pr\left(#1\right)}}
\providecommand{\brak}[1]{\ensuremath{\left(#1\right)}}
\providecommand{\cbrak}[1]{\ensuremath{\left\{#1\right\}}}
\providecommand{\abs}[1]{\left\vert#1\right\vert}
\providecommand{\mean}[1]{E\left[ #1 \right]}

\newcommand{\myvec}[1]{\ensuremath{\begin{pmatrix}#1\end{pmatrix}}}
\newcommand{\mydet}[1]{\ensuremath{\begin{vmatrix}#1\end{vmatrix}}}

\let\vec\mathbf

\renewcommand{\thefigure}{\arabic{table}}
\renewcommand{\thetable}{\arabic{table}}                                     
                               
\title{AI1110 \\ Assignment 15}
\author{Ankit Saha \\ AI21BTECH11004}
\date{11 June 2022}


\begin{document}
	% The title page
	\begin{frame}
		\titlepage
	\end{frame}
	
	% The table of contents
	\begin{frame}{Outline}
    		\tableofcontents
	\end{frame}
	
	% The question
	\section{Question}
	\begin{frame}{Papoulis Exercise 9-21}
	Show that if $X(t)$ is a stationary process with derivative $X'(t)$, then for a given $t$ the random variables $X(t)$ and $X'(t)$ are orthogonal and uncorrelated.
	\end{frame}
	
	% The proof
	\section{Assumptions}
	\begin{frame}{Assumptions}
	\begin{enumerate}
		\item The derivative $X'(t)$ of a wide sense stationary process $X(t)$ is also wide sense stationary, i.e., its mean is constant and its autocorrelation depends only on $\tau = t_1 - t_2$
		\begin{align}
			\mean{X'(t+\tau)X'(t)} = R_{x'x'}(\tau) \qquad \forall t
		\end{align}
		
		\item $X(t)$ and $X'(t)$ are jointly wide sense stationary, i.e., they are both wide sense stationary and their cross-correlation depends only on $\tau = t_1 - t_2$
		\begin{align}
			\mean{X(t+\tau)X'(t)} = R_{xx'}(\tau) \qquad \forall t
		\end{align}
	\end{enumerate}
	\end{frame}	
	
	\section{Proof}
	\begin{frame}{Proof}
	We know that the mean of a stationary process is constant, i.e.,
	\begin{align}
		\mean{X(t)} = \eta_{x}(t) = \eta
	\end{align}
	
	The mean of $X'(t)$ is thus given by
	\begin{align}
		\mean{X'(t)} &= \eta_{x'}(t) \\
		&= \frac{\mathrm{d}}{\mathrm{d}t} \mean{X(t)} \\
		&= \frac{\mathrm{d}}{\mathrm{d}t} \eta \\
		&= 0
	\end{align}
	\end{frame}
	
	\begin{frame}
	The autocorrelation of a wide sense stationary process $X(t)$ depends only on $\tau = t_1 - t_2$
	\begin{align}
		R_{xx}(t_1, t_2) &= \mean{X(t_1) X(t_2)} \\
		&= \mean{X(t_2+\tau) X(t_2)}
	\end{align}
	This is independent of $t_2$, i.e.,
	\begin{align}
		\implies R_{xx}(t+\tau,t) = \mean{X(t+\tau) X(t)} = R_{xx}(\tau) \qquad \forall t \label{autocorr}
	\end{align}	
	Now, 
	\begin{align}
		R_{xx}(-\tau) &= \mean{X(t-\tau) X(t)} \\
		&= \mean{X(t) X(t-\tau)} \\
		&= \mean{X((t-\tau)+\tau) X(t-\tau)} \\
		&= \mean{X(t'+\tau) X(t')} && \text{where } t' = t - \tau \\
		&= R_{xx}(\tau) && \text{from } \ref{autocorr}
	\end{align}
	\end{frame}
	
	\begin{frame}
	On differentiating with respect to $\tau$, we get
	\begin{align}
		R_{xx}'(\tau) &= -R_{xx}'(-\tau) \\
		\implies R_{xx}'(\tau) + R_{xx}'(-\tau) &= 0
	\end{align}
	Substituting $\tau = 0$, we get
	\begin{align}
		R_{xx}'(0) = 0
	\end{align}
	For jointly wide sense stationary processes $X(t)$ and $X'(t)$, 
	\begin{align}
		R_{xx'}(t+\tau,t) &= \mean{X(t+\tau)X'(t)} \\
		&= R_{xx'}(\tau) \qquad \forall t
	\end{align}
	For a given $t$, $X(t)$ and $X'(t)$ are orthogonal if 
	\begin{align}
		R_{xx'}(t,t) &= 0 \\
		\implies R_{xx'}(t+0,t) &= 0 \\
		\implies R_{xx'}(0) &= 0 && \because \tau = t - t = 0 
	\end{align}
	\end{frame}
	
	\begin{frame}
	Now,
	\begin{align}
		R_{xx'}(t_1, t_2) &= \mean{X(t_1)X'(t_2)} \\
		&= \frac{\partial}{\partial t_2} \mean{X(t_1)X(t_2)} \\
		&= \frac{\partial}{\partial t_2} R_{xx}(t_1,t_2)
	\end{align}
	For $\tau = t_1 - t_2$
	\begin{align}
		R_{xx'}(\tau) = \frac{\mathrm{d}R_{xx}(\tau)}{\mathrm{d}\tau}  \frac{\partial \tau}{\partial t_2} = R_{xx}'(\tau) \frac{\partial \tau}{\partial t_2} = -R_{xx}'(\tau)
	\end{align}
	Thus, 
	\begin{align}
		R_{xx'}(0) = -R_{xx}'(0) = 0
	\end{align}

	\begin{alertblock}{Therefore,}
		$X(t)$ and $X'(t)$ are orthogonal
	\end{alertblock}
	\end{frame}

	\begin{frame}
	Now, the cross-covariance of jointly wide sense stationary processes $X(t)$ and $X'(t)$ only depends on $\tau = t_1-t_2$ and is given by
	\begin{align}
		C_{xx'}(t_1, t_2) &= R_{xx'}(t_1, t_2) - \eta_{x}(t_1) \eta_{x'}(t_2) \\
		\implies C_{xx'}(\tau) &= R_{xx'}(\tau) - \eta_{x} \eta_{x'}  
	\end{align}
	For a given $t$, $t_1 = t_2 = t$ and $\tau = t_1 - t_2 = 0$
	\begin{align}
		C_{xx'}(0) &= R_{xx'}(0) - \eta \cdot 0 \\
		&= 0 - 0 \\
		&= 0
	\end{align}
	
	\begin{alertblock}{Therefore,}
		$X(t)$ and $X'(t)$ are uncorrelated
	\end{alertblock}
	\end{frame}
	
\end{document}