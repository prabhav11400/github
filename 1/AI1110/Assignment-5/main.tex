\documentclass[journal,12pt,twocolumn]{IEEEtran}

\usepackage{enumitem}
\usepackage{tfrupee}
\usepackage{amsmath}
\usepackage{amssymb}
\usepackage{gensymb}
\usepackage{graphicx}
\usepackage{txfonts}

\def\inputGnumericTable{}

\usepackage[latin1]{inputenc}                                 
\usepackage{color}                                            
\usepackage{array}                                            
\usepackage{longtable}                                        
\usepackage{calc}                                             
\usepackage{multirow}                                         
\usepackage{hhline}                                           
\usepackage{ifthen}
\usepackage{caption} 
\captionsetup[table]{skip=3pt}  
\providecommand{\pr}[1]{\ensuremath{\Pr\left(#1\right)}}
\providecommand{\cbrak}[1]{\ensuremath{\left\{#1\right\}}}
\renewcommand{\thefigure}{\arabic{table}}
\renewcommand{\thetable}{\arabic{table}}                                     
                               
\title{Assignment 5 \\ \Large AI1110: Probability and Random Variables \\ \large Indian Institute of Technology Hyderabad}
\author{Ankit Saha \\ \normalsize AI21BTECH11004 \\ \vspace*{20pt} \normalsize  26 April 2022 \\ \vspace*{20pt} \Large CBSE Probability Grade 10}


\begin{document}
	% The title
	\maketitle
	
	% The question
	\textbf{Exercise 15.1.10} 
	A piggy bank contains hundred $50$ p coins, fifty \rupee~$1$ coins, twenty \rupee~$2$ coins and ten \rupee~$5$ coins. If it is equally likely that one of the coins will fall out when the bank is turned upside down, what is the probability that the coin:
	\begin{enumerate}[label=(\roman*)]
	\item will be a $50$ p coin?
	\item will not be a \rupee~$5$ coin?
	\end{enumerate}	 
	
	% The answer
	\textbf{Solution.}
	Let a random variable $X \in \cbrak{0,1,2,3}$ denote the possible outcomes of obtaining a random coin from the piggy bank.
	\begin{table}[ht!]
		\centering
		\begin{tabular}{|l|c|c|}

\hline
\textbf{Parameter} & \textbf{Symbol/Formula} & \textbf{Value} \\
\hline
Total investment & $P$ & $4500$ \\
\hline
Face value of a share & $F$ & $100$ \\
\hline
Discount on shares & $d$ & $10$ \\
\hline
Dividend & $D$ & $7.5$ \\
\hline
Number of shares & $N = \dfrac{100P}{F(100-d)}$ & ??? \\
\hline
Annual income & $A = \dfrac{PD}{100-d}$ & ??? \\
\hline

\end{tabular}
		\caption{}
		\label{table:table1}	
	\end{table}
	
	\begin{enumerate}[label=(\roman*)]
	\item The probability that the coin will be a $50$ p coin is given by:
	\begin{align}
	\pr{X=0} &= \frac{n(X=0)}{\sum_{i=0}^3 n(X=i)} = \frac{100}{180} \\
	\therefore \pr{X=0} &= \frac{5}{9} \approx 0.556
	\end{align}

	\item The probability that the coin will not be a \rupee~$5$ coin is given by:
	\begin{align}
	\pr{X \ne 3} &= \frac{\sum_{i \ne 3} n(X=i)}{\sum_{i=0}^3 n(X=i)} = \frac{170}{180} \\
	\therefore \pr{X \ne 3} &= \frac{17}{18} \approx 0.944
	\end{align}	
	\end{enumerate}
	
	\begin{figure}[!ht]
		\centering
		\includegraphics[width=\columnwidth]{figs/fig-1.png}
		\caption{Plot of the probability mass function}
		\label{fig1}
	\end{figure}
	
\end{document}