\documentclass[journal,12pt,twocolumn]{IEEEtran}

\usepackage{enumitem}
\usepackage{tfrupee}
\usepackage{amsmath}
\usepackage{amssymb}
\usepackage{gensymb}
\usepackage{graphicx}
\usepackage{txfonts}

\def\inputGnumericTable{}

\usepackage[latin1]{inputenc}                                 
\usepackage{color}                                            
\usepackage{array}                                            
\usepackage{longtable}                                        
\usepackage{calc}                                             
\usepackage{multirow}                                         
\usepackage{hhline}                                           
\usepackage{ifthen}
\usepackage{caption} 
\captionsetup[table]{skip=3pt}  
\providecommand{\pr}[1]{\ensuremath{\Pr\left(#1\right)}}
\providecommand{\cbrak}[1]{\ensuremath{\left\{#1\right\}}}
\renewcommand{\thefigure}{\arabic{table}}
\renewcommand{\thetable}{\arabic{table}}                                     
                               
\title{Assignment 9 \\ \Large AI1110: Probability and Random Variables \\ \large Indian Institute of Technology Hyderabad}
\author{Ankit Saha \\ \normalsize AI21BTECH11004 \\ \vspace*{20pt} \normalsize  7 May 2022 \\ \vspace*{20pt} \Large CBSE Probability Grade 12}


\begin{document}
	% The title
	\maketitle
	
	% The question
	\textbf{Example 10} 
	A die is thrown. If $E$ is the event `the number appearing is a multiple of three' and $F$ be the event `the number appearing is even' then find whether $E$ and $F$ are independent.	 
	
	% The answer
	\textbf{Solution.}
	Let a random variable $X \in \mathcal{X}$ where $\mathcal{X} = \cbrak{1,2,3,4,5,6}$ denote the number appearing on the die.
	\begin{table}[ht!]
		\centering
		\begin{tabular}{|l|c|c|}

\hline
\textbf{Parameter} & \textbf{Symbol/Formula} & \textbf{Value} \\
\hline
Total investment & $P$ & $4500$ \\
\hline
Face value of a share & $F$ & $100$ \\
\hline
Discount on shares & $d$ & $10$ \\
\hline
Dividend & $D$ & $7.5$ \\
\hline
Number of shares & $N = \dfrac{100P}{F(100-d)}$ & ??? \\
\hline
Annual income & $A = \dfrac{PD}{100-d}$ & ??? \\
\hline

\end{tabular}
		\caption{}
		\label{table:table1}	
	\end{table}	
	\begin{align}
		\pr{E} &= \pr{X \in \cbrak{3,6}} \\
		&= \frac{n(X \in \cbrak{3,6})}{n(X \in \mathcal{X})} \\
		&= \frac26 \\
		&= \frac13 \\
		\pr{F} &= \pr{X \in \cbrak{2,4,6}} \\
		&= \frac{n(X \in \cbrak{2,4,6})}{n(X \in \mathcal{X})} \\
		&= \frac36 \\
		&= \frac12 \\
		\pr{E} \pr{F} &= \frac13 \times \frac12 \\
		&= \frac16		
	\end{align}
	
	Now, the probability of their intersection is given by:
	\begin{align}
		\pr{EF} &= \pr{X \in \cbrak{3,6}, X \in \cbrak{2,4,6}} \\
		&= \pr{X = 6} \\
		&= \frac{n(X = 6)}{n(X \in \mathcal{X})} \\ 
		&= \frac16
	\end{align}
	
	Clearly, 
	\begin{align}
		\pr{EF} = \pr{E} \pr{F}
	\end{align}
	
	Therefore, $E$ and $F$ are independent events.
			
\end{document}