%%%%%%%%%%%%%%%%%%%%%%%%%%%%%%%%%%%%%%%%%%%%%%%%%%%%%%%%%%%%%%%
%
% Welcome to Overleaf --- just edit your LaTeX on the left,
% and we'll compile it for you on the right. If you open the
% 'Share' menu, you can invite other users to edit at the same
% time. See www.overleaf.com/learn for more info. Enjoy!
%
%%%%%%%%%%%%%%%%%%%%%%%%%%%%%%%%%%%%%%%%%%%%%%%%%%%%%%%%%%%%%%%


% Inbuilt themes in beamer
\documentclass{beamer}

% Theme choice:
\usetheme{CambridgeUS}

% Title page details: 
\title{Assignment 9} 
\author{PRABHAV SINGH (BT21BTECH11004)}
\date{30 May 2022}
\logo{\large \LaTeX{}}

\begin{document}
	
	% Title page frame
	\begin{frame}
		\titlepage 
	\end{frame}
	
	% Remove logo from the next slides
	\logo{}
	
	
	% Outline frame
	\begin{frame}{Outline}
		\tableofcontents
	\end{frame}
	
	% Lists frame
	\section{Question}
	\begin{frame}{Question}
	Suppose the Conditional distribution of $ x $ given $ y = n $ is binomial with parameters n and 
	$ p_{1} $. Further, $  Y $ is a binomial random variable with parameters $  M $ and $ p_{2} $. Show that the 
	distribution of $  x  $ is also binomial. Find its parameters.  \\
	\end{frame}
	
	
	% Blocks frame
	\section{Solution}
	\begin{frame}{Solution}
	Using the symbols in their Standard  definition and given as in question ,
	$ 	P(X=k|Y=n) =\binom{n}{k}{p_{1}}^{k}{q_{1}}^{n-k}; k=1,2,3,....n $  \\
	\begin{align}
		E(e^{j{\omega}k}|Y=n)&=\sum_{k=0}^{n}e^{j{\omega}k}P(X=k|Y=n)\\
		&={(p_{1}e^{j{\omega}}+q_{1})}^{n}
	\end{align}
Also we know that ,
\begin{align}
	{\phi}_{x}({\omega})&=E[e^{j{\omega}X}]\\
	&=E(E[e^{j{\omega}x}|Y=n])\\
	&=\sum_{n=0}^{M}E(E[e^{j{\omega}x}|Y=n])P(Y=n)
	\end{align}
\end{frame} 
	
	\begin{frame}
		\begin{align}
			&=\sum_{n=0}^{{\infty}}{(p_{1}e^{j{\omega}}+q_{1})}^{n}\binom{M}{n}{p_{2}}^{n}{q_{2}}^{M-n}\\
			&=\sum_{n=0}^{M}\binom{M}{n}{(p_{2}(p_{1}e^{j{\omega}}+q_{1}))}^{n}{q_{2}}^{M-n}\\
			&= {(p_{1}p_{2}e^{j{\omega}}+q_{1}p_{2}+q_{2})}^{M}
		\end{align}
	But 
	\begin{align}
		(1-p_{1}p_{2} )&= 1- (1-q_{1})(1-q_{2})\\
		&= q_{1}p_{2}+q_{2}
	\end{align}
	Hence 
	{ \textbf{$ {\phi}_{{\omega}}={(pe^{j{\omega}}+q)}^{m} $}} where $ p=p_{1}p_{2} $ \\
	Thus $ \boxed{X \backsim Binomial(M,p_{1}p_{2})} $
\end{frame}
\end{document}